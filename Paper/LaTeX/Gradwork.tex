% Generated on 2023/10/16 15:02:38
\documentclass[12pt,a4paper]{report}
\usepackage[utf8]{inputenc}
\usepackage[explicit]{titlesec}
\usepackage[dvipsnames]{xcolor}
\usepackage{amsmath}
\usepackage{amsfonts}
\usepackage{url}
\usepackage{amssymb}
\usepackage{makeidx}
\usepackage{graphicx}
\usepackage{tikz}
\usepackage[many]{tcolorbox}
\usepackage{standalone}
\usepackage{animate}
\tcbuselibrary{listings}
\usepackage{tabularx}
\usepackage{colortbl}
\usepackage{multicol}
\usepackage{makecell}
\usepackage{adjustbox}
\usepackage{subfig}
\usepackage{lipsum}

\usepackage{framed}
\usepackage{lastpage}


\usepackage{fancyhdr}
\usetikzlibrary{calc,arrows.meta,backgrounds}

\newcommand{\msymbol}[1]{\ifmmode #1 \else $#1$\fi}

\newcommand{\daemathvariable}[1]{{\normalsize \color[RGB]{100,149,237}#1}}
\newcommand{\daemathvalue}[1]{{\normalsize \color[RGB]{153,153,255}#1}}
\newcommand{\daemath}[1]{{\large \color[RGB]{0,0,0}#1}}
\newcommand{\daered}[1]{{\normalsize \color[RGB]{234,0,33}#1}}
\newcommand{\daegreen}[1]{{\normalsize \color[RGB]{0,255,65}#1}}
\newcommand{\mmathvariable}[1]{{\begingroup  \color[RGB]{100,149,237}#1 \endgroup} }
\newcommand{\mmathvalue}[1]{{\begingroup  \color[RGB]{153,153,255}#1 \endgroup} }
\newcommand{\mmath}[1]{{\begingroup  \color[RGB]{0,0,0}#1 \endgroup} }
\newcommand{\mred}[1]{{\begingroup  \color[RGB]{234,0,33}#1 \endgroup} }
\newcommand{\mgreen}[1]{{\begingroup  \color[RGB]{0,255,65}#1 \endgroup} }
\definecolor{backgroundStroke}{RGB}{0,0,0}
\definecolor{backgroundFill}{RGB}{255,255,255}
\definecolor{backgroundText}{RGB}{0,0,0}
\definecolor{bgChapter}{RGB}{91,155,213}
\definecolor{bgSection}{RGB}{222,234,246}
\definecolor{fontSubsection}{RGB}{53,124,177}
\colorlet{shadecolor}{bgChapter}

\renewcommand{\contentsname}{Contents}
\newcommand{\pbreak}{\vskip 0.5cm}
\renewcommand{\thesection}{\arabic{section}}

\newcommand{\daechapter}[1]{
  \chapter*{#1} % Create an unnumbered chapter
  \addcontentsline{toc}{chapter}{#1} % Add the chapter to the ToC
}

\newcommand{\daesection}[1]{
  \section*{#1} % Create an unnumbered chapter
  \addcontentsline{toc}{chapter}{#1} % Add the chapter to the ToC
}

\newtcolorbox{ybox}{
  colback=yellow,
  colframe=yellow,
  boxrule=0pt,
  arc=0pt,
  outer arc=0pt,
  boxsep=0pt,
  left=1pt,
  right=1pt,
  top=1pt,
  bottom=1pt
}

\newtcolorbox{gbox}{
  colback=lightgray,
  colframe=lightgray,
  boxrule=0pt,
  arc=0pt,
  outer arc=0pt,
  boxsep=0pt,
  left=1pt,
  right=1pt,
  top=1pt,
  bottom=1pt
}



\pagecolor{backgroundFill}
\color{backgroundText}
\everymath{\color{backgroundText}}

\titleformat{\chapter}[display]{\normalfont\color{white} \begin {shaded*}\bfseries}{\large\chaptername~\thechapter}{20pt}{\Large#1\end{shaded*}}

\titleformat{\section}
  {\normalfont\Large\bfseries} 
  {}{0em} 
  { 
    \colorbox{bgSection}{%
      \parbox{\dimexpr\textwidth-2\fboxsep\relax}{\thesection\quad#1}
    }
  }
  
\titleformat{\subsection}
  {\color{fontSubsection}\titlerule\vspace{1ex} \normalfont\large\bfseries}
  {\thesubsection}{1em}
  {
  	\quad #1
  }
  
\titleformat{\subsubsection}
  {\color{fontSubsection}\titlerule\vspace{1ex}\normalfont\normalsize\bfseries}
  {\thesubsubsection}{1em}
  {	
  	 \quad #1   	 
  }
 
% information about the gradwork
\def\title{The title of your gradwork}
\def\subtitle{Subtitle}
\def\author{Your name}
\def\academicyear{2024-2025}


\fancyhf{}
\fancyhead[L]{\author}
\fancyfoot[L]{DAE - Graduation work 2023-2024}
\fancyfoot[R]{\thepage/\pageref{LastPage}}


\begin{document}
\pagestyle{fancy}




\begin{titlepage}
   \begin{center}
       \vspace*{1cm}
       \Huge{\title}
       \vspace{0.5cm}
       
       \Large{\subtitle}
       \vspace{1.5cm}
       
       \textbf{\author}
       \vfill
       \Large{Graduation work \academicyear}
      
     
		\begin{figure}[h]
    		\begin{minipage}[t]{0.40\textwidth}
        		\includegraphics[height=2cm]{logos/Howest_logo.png}
    		\end{minipage}\hfill
	    	\begin{minipage}[t]{0.40\textwidth}
    		    \centering
		        \includegraphics[height=2cm,trim=2cm 7cm 0.5cm 6cm,clip]{logos/DAE_logo.pdf}
       
    		\end{minipage}
			\vspace{0.8cm}
		\end{figure}
   \end{center}
\end{titlepage}

\tableofcontents

\daechapter{Abstract {\&} Keywords}

\begin{ybox}
An abstract explains the outline of the paper concisely (the methods, results, etc.). Maximum length of 250 words, preferably both in English and Dutch.
\end{ybox}

\begin{gbox}
\lipsum[1-1]
\end{gbox}

\daechapter{Preface}

\begin{ybox}
A preface is a statement of the author's reasons for undertaking the work and may include personal comments that are not directly relevant to other sections of the thesis or dissertation. No word count limit.
\end{ybox}

\begin{gbox}
\lipsum[1-1]
\end{gbox}

\daechapter{List of figures}
\addcontentsline{toc}{chapter}{List of figures}

\begin{ybox}
The list of figures lists the figures in the order in which they appear throughout the thesis. They may be numbered sequentially, or be subdivided following the chapters in which they appear.
\end{ybox}

\begin{ybox}
Figure 1: A picture showing something
\end{ybox}

\begin{ybox}
Figure 2: A graph showing another thing
\end{ybox}

\begin{ybox}
Figure 3.1: A tabel showing yet another thing, that appears in chapter 3.
\end{ybox}

\daechapter{Introduction}
\addcontentsline{toc}{chapter}{Introduction}

\begin{ybox}
In the introduction, you write the background of your topic and discuss the observation that spurred you on to do this research project. Explain the purpose of the paper and present your research question(s) and the hypothesis at the end of this section. This section is typically a couple of pages long.
\end{ybox}
\pbreak
\begin{gbox}
\lipsum[1-1]
\end{gbox}
\pbreak
\begin{gbox}
\lipsum[2-2]
\end{gbox}
\pbreak
\begin{gbox}
Math symbols, method to use math symbol inside or outside of math text with the msymbol command:
\msymbol{\alpha},\msymbol{\beta}
\end{gbox}

\daechapter{Literature study / Theoretical framework}

\begin{ybox}
In the literature review, you present the secondary research you have conducted. You detail the background of your topics and write about the concepts that are relevant to the study. Assume that not every reader has the same skillset or -level as you do! This section typically requires a substantial amount of references and can be a lengthy section that requires a considerable amount of pages.
\end{ybox}

\begin{gbox}
\lipsum[1-1]
\end{gbox}

\daechapter{Research}

\begin{ybox}
In the research section, you detail the elements of your experiment(s), the tests, objects you will test upon and subjects you will test with, the data gathering, data cleaning or feature extraction, measurements, … and you present the results obtained in an objective manner for each of the tests you conducted.
\end{ybox}

\begin{gbox}
\lipsum[1-1]
\end{gbox}

\clearpage
\section{Topic 1}

\subsection{Sub Topic 1}

\begin{gbox}
\lipsum[1-1]
\end{gbox}

\subsubsection{Sub sub Topic 1}

\begin{gbox}
\lipsum[1-1]
\end{gbox}


\section{Topic 2}

\subsection{Sub Topic 1}

\begin{gbox}
\lipsum[1-1]
\end{gbox}

\subsubsection{Sub sub Topic 1}

\begin{gbox}
\lipsum[1-1]
\end{gbox}

\daechapter{Case Study}

\begin{ybox}
Alternatively, as opposed to research, you might have opted for a case study. Whichever you choose, you detail the elements of your experiment(s), the tests, objects you will test upon and subjects you will test with, the data gathering, data cleaning or feature extraction, measurements, … and you present the results obtained in an objective manner for each of the tests you conducted.
\end{ybox}

\begin{gbox}
\lipsum[1-1]
\end{gbox}

\section{Introduction}

\begin{ybox}
In the introduction, you write the background of your topic, explain the purpose of the paper more broadly, and explain the hypothesis, and the research question(s).
\end{ybox}

\begin{gbox}
\lipsum[1-1]
\end{gbox}

\section{Modelling}

\subsection{Blockout}

\begin{gbox}
\lipsum[1-1]
\end{gbox}

\begin{figure}[h]
  \centering
    
      \includegraphics[trim=2cm 7cm 0.5cm 6cm,width=1.0\linewidth]{logos/DAE_logo.pdf}
  \caption{The dae logo}
\end{figure}

\begin{gbox}
\lipsum[1-1]
\end{gbox}

\subsection{ZBrush}

\begin{gbox}
\lipsum[1-1]
\end{gbox}

\section{Texturing}

\begin{gbox}
\lipsum[1-1]
\end{gbox}

\section{Shading}

\begin{gbox}
\lipsum[1-1]
\end{gbox}

\section{Lighting}

\begin{gbox}
\lipsum[1-1]
\end{gbox}

\daechapter{Discussion}

\begin{ybox}
In this section, you offer an interpretation of the results you obtained and try to relate them to the theoretical framework you presented. This is typically not a very long section, but obviously one of the most important ones.
\end{ybox}

\begin{gbox}
\lipsum[1-1]
\end{gbox}
\daechapter{Conclusion}

\begin{ybox}
In this section, you ascertain the demonstrable outcomes of your study and outline the merits of the project for the academic field and the discourse community. This is typically not a very long section, but obviously also one of the more important ones.
\end{ybox}

\begin{gbox}
\lipsum[1-1]
\end{gbox}

\daechapter{Future work}

\begin{ybox}
This section is sometimes standalone, sometimes incorporated in the conclusion. It looks at the shortcomings of the study, alternative strategies, and what could be the next course of action in the research field. This is typically not a very long section.
\end{ybox}

\begin{gbox}
\lipsum[1-1]
\end{gbox}

\daechapter{Critical Reflection}

\begin{ybox}
This section is typically associated with a bachelor paper, not other forms of serious writing. It allows the student to reflect on the learning outcomes, both academically and in terms of personal growth.
\end{ybox}

\begin{gbox}
\lipsum[1-1]
\end{gbox}

\daechapter{References}

\begin{ybox}
In this section, you list all the references you made in alphabetical order; consequently adhere to the referencing style you have chosen.
\end{ybox}

Casey Raes (2014), Processing (second edition). Saccade. (n.d.). In Wikpedia. Retrieved November 6 2016 from https://en.wikipedia.org/wiki/Saccade

Sarah Northway (2016) A year in VR Northway [Powerpoint slides] from https://www.gdcvault.com/play/1024631/A-Year-in-VR-A

\end{document}
